On montre ici brièvement la syntaxe d'un fichier ATL avec quelques exemples :

\lstset{
   language={}
}
\begin{lstlisting}
-- Traduire un Process en un PetriNet de même nom
rule Process2PetriNet {
   from p: simplepdl!Process
   to pn: petrinet!PetriNet (name <- p.name)
}
\end{lstlisting}

Les règles de transformation sont exprimées avec le mot-clé "rule".
Les mots "from" et "to" permettent d'identifier les éléments de la transformation.


\begin{lstlisting}
-- Traduire une WorkSequence en un motif sur le réseau de Petri
rule WorkSequence2PetriNet {
   from ws: simplepdl!WorkSequence
   to
      a_ws: petrinet!Arc(
         petriNet <- ws.getProcess(),
         readOnly <- true,
         multiplicity <- 1,
         predecessor <- thisModule.resolveTemp(ws.predecessor,
            if((ws.linkType = #finishToStart) or (ws.linkType = #finishToFinish))
               then 'p_finished'
               else 'p_started'
            endif),
         successor <- thisModule.resolveTemp(ws.successor,
            if((ws.linkType = #finishToStart) or (ws.linkType = #startToStart))
               then 't_start'
               else 't_finish'
            endif)
         )
}
\end{lstlisting}

La commande "resolveTemp" permet d'identifier un élément parmi ceux présents dans les règles de transformations.
