Xtext permet de spécifier la syntaxe d'un fichier décrivant un modèle.
Cette spécification permet également la construction automatique du métamodèle correspondant à la syntaxe concrète décrite.\\

En fait, il apparait que le métamodèle construit est légèrement différents du métamodèle initial.
Ceci est d'autant plus vrai que la syntaxe est simplifiée et s'éloigne de la structure du modèle.
Ce n'est pas si grave, puisque l'on connait maintenant les outils permettant de faire des transformations M2M.\\

Voilà à quoi ressemble la syntaxe d'un fichier xtext, ici le contenu du xtext pour petrinet :

\lstset{
   language={}
}
\begin{lstlisting}
PetriNet :
   'petrinet' name=ID '{'
      petriNetElements+=PetriNetElement*
   '}'
   ;

PetriNetElement :
   Node
   | Arc
   ;

Node :
   Place
   | Transition
   ;

Place :
   'place' name=ID ('('marking=INT')')?
   ;

Transition :
   'transition' name=ID
   ;

Arc :
   'arc' ('('multiplicity=INT')')? (readOnly ?= 'r')?
   'from' predecessor=[Node]
   'to' successor=[Node]
   ;
\end{lstlisting}

Le fichier xtext pour notre syntaxe pdlx (pdl extended for ressources) est donnée dans les sources.\\

On donne ici un exemple en .pdlx :

\begin{lstlisting}
process test {
   wd conception
   wd redactionDoc
   wd developpement
   wd redactionTests

   ws s2s from conception to redactionDoc
   ws f2f from conception to redactionDoc
   ws f2s from conception to developpement
   ws s2s from conception to redactionTests
   ws f2f from developpement to redactionTests

   r concepteur (3)
   r developpeur (2)
   r machine (4)
   r redacteur (1)
   r testeur (2)

   set set_conception wd conception
      n concepteur(2) set set_conception
      n machine(2) set set_conception
   set set_redactionDoc wd redactionDoc
      n machine(1) set set_redactionDoc
      n redacteur(1) set set_redactionDoc
   set set_developpement wd developpement
      n developpeur(2) set set_developpement
      n machine(3) set set_developpement
   set set_redactionTests wd redactionTests
      n testeur(2) set set_redactionTests
      n machine(2) set set_redactionTests
}
\end{lstlisting}

Et pour un fichier en .petri :

\begin{lstlisting}
petrinet petrinet1 {
   place place1
   place place2 (2)
   transition transition1
   arc r from place1 to transition1
   arc (2) from transition1 to place2
}
\end{lstlisting}
